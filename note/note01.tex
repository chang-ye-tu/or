\documentclass[11pt]{extarticle} 
\usepackage{unicode-math}
\usepackage{mathrsfs}
\usepackage{amsthm,graphicx,xcolor,natbib,enumitem}
\usepackage[paperwidth=126mm, paperheight=96mm, top=5mm, bottom=5mm, right=5mm, left=5mm]{geometry}
\pagenumbering{gobble}

\usepackage[BoldFont,SlantFont]{xeCJK}  
\xeCJKsetemboldenfactor{2}
\setCJKmainfont{cwTeX Q Yuan Medium}

\usepackage{hyperref}
\hypersetup{
    colorlinks,
    linkcolor={red!50!black},
    citecolor={blue!60!black},
    urlcolor={blue!60!black}
    %urlcolor={blue!80!black}
}

\begin{document}
\title{\texorpdfstring{\vspace{15mm} Operations Research\\ 01. Orientation and Introduction}{Operations Research\\ 01. Orientation and Introduction}}
\author{}
%\date{September 12, 2023}
\date{}
\maketitle
\newpage

%\section*{評分標準}
%
%學期總成績基本分為 75 分,可選擇執行以下項目加分:
%\begin{itemize}[itemsep=0pt]
%  \item 課前問卷 $1$ 次,滿分 $5$ 分,得分 $q$ 分
%  \item 習題(理論)$a$ 次,各次滿分 $v_i$ 分,得分 $\alpha_i$ 分
%  \item 習題(程式)$b$ 次,各次滿分 $w_j$ 分,得分 $\beta_j$ 分
%  \item 考試(紙筆)$c$ 次,各次滿分 $x_k$ 分,得分 $\gamma_k$ 分
%  \item 考試(程式)$d$ 次,各次滿分 $y_l$ 分,得分 $\delta_l$ 分
%  \item 報告(書面 + 上台)$e$ 次,各次滿分 $z_m$ 分,得分 $\varepsilon_m$ 分
%\end{itemize}
%學期總成績最後分數為 $$75 + q + \sum_{i=1}^a\alpha_i + \sum_{j=1}^b\beta_j + \sum_{k=1}^c\gamma_k + \sum_{l=1}^d\delta_l + \sum_{m=1}^e\varepsilon_m$$ 超出 99 分以 99 分計。 
%
%\newpage
%
%\section*{加分項目說明}
%
%\begin{itemize}[itemsep=0pt]
%  \item 課前問卷:機率、線性代數、微積分與 Python 基礎知識。
%  \item 習題:理論習題或程式習題。
%    \begin{itemize}
%      \item 習題繳交後會公佈答案並完整檢討。
%    \end{itemize}
%  \item 考試:紙筆作答考試或程式考試。
%    \begin{itemize}
%      \item 紙筆作答考試內容為理論習題選輯,閉書考試。
%      \item 程式考試必須在 48 小時內完成指定題目並寄回 Jupyter Notebook 檔案之答案。
%    \end{itemize}
%  \item 報告:事先選定主題之書面報告,佐以上台演示 15 分鐘及提問回答。
%    \begin{itemize}
%      \item 書面報告與投影片禁止使用 Word\textsuperscript{\textregistered} 與 PowerPoint\textsuperscript{\textregistered} 製作,必須使用課堂上講授之 \LaTeX{} 與 Markdown 語法配合製作 Jupyter Notebook 等開源格式文件。
%    \end{itemize}
%\end{itemize}
%
%\newpage
%
\section*{課程內容}

\begin{itemize}[itemsep=0mm]
  \item Convex Optimization
    \vspace{-2mm}
    \begin{itemize}
      \item Convexity and Duality
      \item Selected Algorithms
    \end{itemize}
  \item Linear Programming
    \vspace{-2mm}
    \begin{itemize}
      \item Various Applications
      \item Simplex Method
    \end{itemize}
  \item Sequential Decision Problems
    \vspace{-2mm}
    \begin{itemize}
      \item Dynamic Programming 
      \item Reinforcement Learning 
    \end{itemize}
  \item Jupyter Notebook: Markdown \& \LaTeX{}
  \item Intermediate Scientific Python
  \item CVXPY \& Applications (能否帶可執行計算之筆電來上課?)
\end{itemize}

\newpage

\section*{Learning Python: Curated Resource List}
\begin{itemize}
  \item Novice 
    \vspace{-2mm}
    \begin{itemize}
      \item \citet{bell,luba,matt}
    \end{itemize}
  \item Intermediate to Advanced 
    \vspace{-2mm}
    \begin{itemize}
      \item \citet{ramalho,martelli,beazley_ess,cookbook,beazley_dis,slatkin}
    \end{itemize}
  \item High Performance Computing 
    \vspace{-2mm}
    \begin{itemize}
      \item \citet{ozsvald,antao}
    \end{itemize}
  \item Miscellaneous
    \vspace{-2mm}
    \begin{itemize}
      \item \citet{lerner}
      \item \href{https://github.com/gto76/python-cheatsheet}{Comprehensive Python Cheat Sheet}
      \item \href{https://github.com/rougier}{Nicholas P. Rougier} 
        \vspace{-1mm}
        \begin{itemize}
          \item \href{https://www.labri.fr/perso/nrougier/from-python-to-numpy/}{From Python to NumPy}
          \item \href{https://github.com/rougier/scientific-visualization-book}{Scientific \& Visualization: Python \& Matplotlib}
        \end{itemize}
    \end{itemize}
\end{itemize}

\newpage

\bibliographystyle{elsarticle-harv}
\bibliography{note01}

\end{document}
