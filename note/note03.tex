\documentclass[11pt]{extarticle} 
\usepackage{unicode-math}
\usepackage{mathrsfs}
\usepackage{amsthm,graphicx,xcolor,natbib,enumitem,booktabs,tabularx}
\usepackage[paperwidth=126mm, paperheight=96mm, top=5mm, bottom=5mm, right=5mm, left=5mm]{geometry}
\pagenumbering{gobble}

\usepackage[BoldFont,SlantFont]{xeCJK}  
\xeCJKsetemboldenfactor{2}
%\setCJKmainfont{cwTeX Q Kai Medium}
%\setCJKmainfont{cwTeX Q Ming Medium}
\setCJKmainfont{cwTeX Q Yuan Medium}
%\usepackage[AutoFakeBold,AutoFakeSlant]{xeCJK}  
%\setCJKmainfont[AutoFakeSlant=.1,AutoFakeBold=1]{cwTeX Q Kai Medium} 
%\setCJKfamilyfont{kaiv}[Vertical=RotatedGlyphs]{cwTeX Q Medium}
%\setmainfont{texgyrepagella-regular.otf}
%\setmathfont{texgyrepagella-math.otf}

\usepackage{hyperref}
\hypersetup{
    colorlinks,
    linkcolor={red!50!black},
    citecolor={blue!60!black},
    urlcolor={blue!60!black}
    %urlcolor={blue!80!black}
}

\newcommand{\ds}{\displaystyle}
\newcommand{\ie}{\,\Longrightarrow\,}
\newcommand{\ifff}{\,\Longleftrightarrow\,}
\newcommand{\mi}{\mathrm{i}}
\DeclareMathOperator*{\dom}{dom}
\DeclareMathOperator*{\codom}{codom}
\DeclareMathOperator*{\ran}{ran}
\newcommand{\floor}[1]{\lfloor #1 \rfloor}
\newcommand{\ceil}[1]{\lceil #1 \rceil}
\newcommand{\Set}[2]{\big\{ \ #1\ \big|\ #2\ \big\}}
\newcommand{\pdiff}[2]{\frac{\partial\hfil#1\hfil}{\partial #2}}

\theoremstyle{definition}
\newtheorem*{dfn}{Definition}
\newtheorem*{prp}{Property}
\newtheorem*{thm}{Theorem}
\newtheorem*{ex}{Example}
\newtheorem*{sol}{Solution}
\newtheorem*{prf}{Proof}

\begin{document}
\title{\texorpdfstring{\vspace{15mm} Operations Research\\ 03. Scientific Python Refresher}{Operations Research\\ 03. Scientific Python Refresher}} 
\author{}
\date{}
\maketitle
\newpage

\section*{Learning Resources}

\begin{itemize}\setlength{\itemsep}{-4pt}
  \item \href{https://www.dabeaz.com/}{David Beazley$^\star$}
    \vspace{-2mm}
    \begin{itemize}
      \item \citet{beazley_ess,beazley_dis,cookbook}
      \item Online Courses
        \begin{itemize}
          \item \href{https://github.com/dabeaz-course/practical-python}{Practical Python Programming}
          \item \href{https://github.com/dabeaz-course/python-mastery}{Advanced Python Mastery}
        \end{itemize}
    \end{itemize}
  \item Cheat Sheets
    \vspace{-2mm}
  \begin{itemize}
    \item \href{https://www.pythoncheatsheet.org/}{Python Cheat Sheet}
    \item \href{https://ipgp.github.io/scientific_python_cheat_sheet/}{Scientific Python Cheatsheet}
    \item \href{https://s3.amazonaws.com/assets.datacamp.com/blog_assets/Numpy_Python_Cheat_Sheet.pdf}{``Python for Data Science'' Cheat Sheet: NumPy Basics}
    \item \href{https://github.com/gto76/python-cheatsheet}{Comprehensive Python Cheat Sheet}
    \item \href{https://sebastianraschka.com/blog/2014/matrix_cheatsheet_table.html}{Sebastian Raschka$^\star$: Matrix Cheatsheet}
    \item \href{https://ehmatthes.github.io/pcc_3e/cheat_sheets/}{Eric Mattes: Cheat Sheets}
    \item \href{https://github.com/kieranholland/best-python-cheat-sheet}{The Best Python Cheat Sheet}
  \end{itemize}
  \item \href{https://www.reddit.com/r/Python/comments/103i4d2/what_are_the_best_books_to_learn_python/}{A Reddit pointer} for best books on (project-based) Python
\end{itemize}

\newpage

\begin{ex}
  In each of the followings, write an oneliner of {\tt Python} to complete the task. 
  \begin{enumerate}\setlength{\itemsep}{-4pt}
    \item {\tt l = ['a', 'b', 'c', 'd', 'e', 'f', 'g', 'h']} $\ie$ \\ {\tt ll = ['b', 'd', 'f', 'h']}
    \item {\tt l = [1, 2, 3, 4, 5]} $\ie$ {\tt ll = [1, 4, 9, 16, 25]}
    \item {\tt l = [2, 5, 5, -1, -1, -1]} $\ie$ {\tt ll = [2, 5, -1]}
    \item {\tt l = [1, 2, 3], ll = ['A', 'B', 'C']} $\ie$ \\{\tt L = [(1, 'A', 1), (2, 'B', 2), (3, 'C', 3)]}
    \item {\tt l = [(1, 2), (2, -2), (5, -3), (-20, 40)]} $\ie$ \\ {\tt ll = [(-20, 40), (1, 2), (5, -3), (2, -2)]}
  \end{enumerate}
\end{ex}
 
\begin{sol}
  \begin{enumerate}[label=(\alph*)]\setlength{\itemsep}{-4pt}
    \item[]
    \item {\tt ll = l[1::2]}
    \item {\tt ll = [i**2 for i in l]}
    \item {\tt ll = list(set(l))}
    \item {\tt L = list(zip(l, ll, l))}
    \item {\tt ll = sorted(l, key=lambda x: sum(x), reverse=True)}
  \end{enumerate}
\end{sol}

\newpage

\begin{ex} 
  Write a {\tt Python} function {\tt mymin(l)} to return the least element of {\tt l}, e.g. {\tt mymin([2, 7, 4, 2]) = 2}, {\tt mymin([-1, 2, -5, 4, 3]) = -5}.
\end{ex}

\begin{sol} Without using the built-in {\tt min} function, \\
  \begin{verbatim}
    def mymin(l):
        a = l[0]
        for i in l[1:]:
            if i <= a:
                a = i
        return a
  \end{verbatim}
\end{sol}

\newpage

\begin{ex} 
  Write a {\tt Python} function {\tt mysqrt(a)} to compute the square root of {\tt a}, e.g. \\{\tt mysqrt(2) = 1.4142135623730951}, \\ {\tt mysqrt(3) = 1.7320508075688772}.
\end{ex}

\begin{sol}  
  Using Newton's method:
  \begin{itemize} 
    \item Find the root of $f(x) = x^2 - a = 0$; update the $n$-th iteration by 
      \begin{align*}
        x_{n + 1} = x_n - \frac{f(x_n)}{f'(x_n)}
      \end{align*}
    \item
  \begin{verbatim}
    def mysqrt(a):
        x = a
        for i in range(100):
            x -= (x ** 2 - a) / (2 * x) 
        return x
  \end{verbatim}
  \end{itemize}
\end{sol}

\newpage
\bibliographystyle{elsarticle-harv}
\bibliography{note03}

\end{document}
